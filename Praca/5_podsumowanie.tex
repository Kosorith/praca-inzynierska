\chapter{Podsumowanie i wnioski}\label{chap:podsumowanie}

\section{Dyskusja wyników}

Dzięki zrealizowaniu pracy sposób wyświetlania rodzin typów w błędach i
ostrzeżeniach kompilatora ujednolicono. Dodane zostało wyświetlanie prawych
stron równań rodzin typów danych w sposób naśladujący składnię tych
konstrukcji. Stary, oddający wewnętrzną reprezentację rodzin typów danych sposób
wyświetlania został zachowany w zrzucie z etapu sprawdzania typów.

Dodane zostało także opcjonalne wyświetlanie ostrzeżeń o nieużywanych zmiennych
w rodzinach typów. Zmienna uważana jest za nieużywaną jeżeli występuje we
wzorcach po lewej stronie równania tylko raz i nie występuje w typie po prawej
stronie.

Poprawiony został błąd z zamianą zmiennych typów zaczynających się od
podkreślnika w symbole wieloznaczne w kontekstach, gdzie są one
niedozwolone. Algorytm renamera został również zmieniony tak, by nie dokonywał
zamiany zmiennych jawnie związanych kwantyfikatorem. Dzięki temu uaktywnienie
rozszerzenia \code{NamedWildCards} nie powoduje już odrzucania programów.

Wszystkie te trzy modyfikacje przeszły proces rewizji kodu w systemie
Phabricator i znalazły się w repozytorium. Od tamtej pory wprowadzony kod
podlegał dodatkowym modyfikacjom i refaktoringowi dokonanemu przez innych
programistów. W szczególności część dotycząca \code{NamedWildCards} została od
tamtego czasu w dużej części zastąpiona przez alternatywne rozwiązanie Simona
Peytona Jonesa. Jednak wszystkie usprawnienia dalej są dostępne w
kompilatorze. Dlatego cele pracy można zatem uznać za zrealizowane, choć
oczywiście praca mogłaby obejmować więcej usprawnień.

\section{Perspektywy dalszych badań}
System Trac na stronie GHC zawiera obecnie ponad 1600 otwartych
zgłoszeń\cite{WikiTickets}. Inne prace mogą podjąć się dokonania innych
usprawnień związanych z programowaniem z użyciem typów lub z inną częścią
kompilatora. Złożone propozycje mają poświęcone sobie podstrony na wiki GHC,
gdzie można znaleźć ich planowane funkcje, projekty, opisy przebiegu
implementacji i odnośniki do prac badawczych, na których bazują. Są to na
przykład propozycja dodania typów zależnych do Haskella lub wprowadzenia
definiowanych przez użytkownika błędów typów. Praca mogłaby polegać również na
stworzeniu nowej propozycji i zaimplementowaniu jej. Możliwości są szerokie i
wiele osób zdecydowało się już poświęcić swój czas GHC.

\todo[disable,inline,size=\tiny]{Podsumowanie jest, obok Wstępu, najważniejszym rozdziałem pracy. Należy tutaj
jeszcze raz podsumować wykonane prace. Szczególny nacisk należy położyć na wkład
własny autora i~uzyskane oryginalne rezultaty. Należy odwołać się do celów pracy
z~rozdziału \ref{sec:cele_pracy} -- można je powtórzyć -- i~jasno wskazać, że
zostały one zrealizowane (należy powołać się na wyniki z~rozdziału
\ref{chap:badania}). Wyniki należy podsumować zwięźle i~precyzyjnie, np.
\textit{uzyskano przyspieszenie algorytmu o~X\%..., skrócono czas o~...}  itd.
Należy wskazać perspektywy dalszych badań bądź zastosowanie uzyskanych
rezultatów do rozwiązania problemów znanych z literatury.}
