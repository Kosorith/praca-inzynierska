\chapter{Płyta CD}\label{app:plyta}

\begin{figure}[htb]
\makebox[\textwidth]{\framebox[12.8cm]{\rule{0pt}{12.8cm}}}
\end{figure}
\pagebreak
{\color{red} Do pracy należy dołączyć podpisaną płytę CD w~papierowej
  kopercie.  Poniżej należy zamieścić opis zawartości katalogów.}

Zawartość katalogów na płycie:
\begin{description}
\item[dat] : {\color{red} pliki z~danymi wykorzystane w~trakcie badań}
\item[db] : {\color{red} Zrzut bazy danych potrzebnej do działania aplikacji}
\item[dist] : dystrybucyjna wersja aplikacji przeznaczona do uruchamiania
\item[doc] : elektroniczna wersja pracy dyplomowej oraz dwie prezentacje
  wygłoszone podczas seminarium dyplomowego
\item[ext] : {\color{red} ten katalog powinien zawierać ewentualne aplikacje
  dodatkowe potrzebne do uruchomienia stworzonej aplikacji, np. środowisko Java,
  PostgreSQL itp.}
\item[src] : kod źródłowy aplikacji {\color{red}(projekt środowiska Eclipse /
  Netbeans / Qt Creator / ... })
\end{description}

{\color{red} Oczywiście nie wszystkie powyższe katalogi są wymagane, np. dat, db
  albo ext mogą być niepotrzebne.}
