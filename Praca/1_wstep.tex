\chapter{Wstęp}

\footnotetext{Pierwotna wersja tego dokumentu powstała w oparciu
  o~\cite{Nie10}.}

Praca dyplomowa \underline{musi} być samodzielnym opracowaniem przez dyplomanta
\emph{wybranego tematu badawczego} pod kierunkiem promotora.  Temat i~zakres
pracy musi wiązać się ze specjalnością, na której studiuje dyplomant.
Orientacyjna objętość pracy inżynierskiej / licencjackiej (I-go stopnia) to
50-80 stron, zaś pracy magisterskiej (II-go stopnia) -- 70-120
stron.

Wstęp rozprawy powinien jasno określać tematykę i~zakres podejmowanego problemu.
Należy wskazać dlaczego dana tematyka została podjęta. Czy rozwiązania
istniejące w~danej dziedzinie nie są wystarczające? Czy problem można rozwiązać
inaczej? Czy podejmowany problem jest aktywnym tematem badawczym? Przed jakimi
wyzwaniami stoi osoba podejmująca tematykę? Na tym etapie należy zarysować
problem w~sposób ogólny.

\section{Cele pracy}\label{sec:cele_pracy}
\comment{Należy jasno określić cele pracy, np.:}

Na podstawie powyższego przeglądu literatury stawiam następujące cele pracy:

\begin{itemize}
 \item Stworzenie aplikacji / metody ...
 \item Wykazanie skuteczności ...
 \item Opis zastosowania technologii X w~problemie Y ...
\end{itemize}

\comment{Cele muszą być sformułowane w~sposób zwięzły i~\textbf{ścisły}.}

\comment{Alternatywnie, zamiast zakładać tutaj cele do realizacji, można
  opisywać wkład pracy dyplomowej w stan wiedzy w danej dziedzinie.  W ten
  sposób czytelnik już na wstępie wie, jakie są osiągnięcia autora.}

\section{Przegląd literatury oraz uzasadnienie wyboru tematu}

W tym podrozdziale należy szczegółowo uzasadnić dlaczego wybrany został taki
a~nie inny temat pracy. Trzeba przede wszystkim zaprezentować aktualny stan
wiedzy w~danej dziedzinie. Oznacza to konieczność omówienia książek
(ew. artykułów naukowych bądź dokumentacji technicznej) z~których będzie się
korzystać w~trakcie rozprawy. Następnie należy wskazać -- tym razem już
konkretnie -- co nowego zamierza się zrobić. Podstawowymi celami tego
podrozdziału jest wprowadzenie czytelnika w~aktualny stand danej dziedziny
i~przekonanie go że \textbf{naprawdę warto zajmować się podjętym tematem}.

\section{Układ pracy}
\comment{Tutaj należy zamieścić opis dalszej zawartości pracy.}

Struktura dalszej części szablonu jest następująca: Rozdział \ref{chap:teoria}
zawiera opis teorii ... Rozdział \ref{chap:nowa_teoria} przedstawia nową teorię
wprowadzoną przez autora pracy ...  Rozdział \ref{chap:badania} przedstawia
wyniki badań / opis stworzonej aplikacji ... Rozdział \ref{chap:podsumowanie}
podsumowuje uzyskane wyniki oraz płynące z~nich wnioski ... W~Dodatku
\ref{app:edycja} zawarto uwagi dotyczące formatowania pracy z~użyciem systemu
\LaTeX. Dodatek \ref{app:plyta} zawiera płytę CD z~aplikacją stworzoną w~ramach
pracy...
